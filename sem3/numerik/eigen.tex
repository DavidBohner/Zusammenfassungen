\documentclass[12pt]{article}

% all the packages we will import
\usepackage[dvipsnames,usenames, table]{xcolor} 
\usepackage[utf8]{inputenc}
\usepackage[T1]{fontenc}
\usepackage[english]{babel}
\usepackage[a4paper, margin=1in]{geometry}
\usepackage{lipsum}
\usepackage{libertine}
\usepackage{amsthm}
\usepackage{amssymb}
\usepackage{amsmath}
\usepackage{libertinust1math}
\usepackage{tikz}
\usepackage{mathtools}
\usepackage{fancyhdr}
\usepackage{tikz}
\usepackage{graphicx}
\usepackage{microtype}
\usepackage{listings}
\usepackage[hidelinks]{hyperref}
\usepackage{tocloft}
\usepackage{lettrine}
\usepackage{GoudyIn}

% commands we are defining / redefining
\renewcommand{\LettrineFontHook}{\GoudyInfamily{}}

\newcommand{\tocr}{
\renewcommand\cftaftertoctitle{\par\noindent\hrulefill\par\vskip-0.65em}
\tableofcontents
\noindent\hrulefill
}
\DeclarePairedDelimiter{\ceil}{\lceil}{\rceil}
\renewcommand{\ttdefault}{pcr}

    
        
\usepackage{inconsolata}

\definecolor{background}{HTML}{fcfeff}
\definecolor{comment}{HTML}{b2979b}
\definecolor{keywords}{HTML}{255957}
\definecolor{basicStyle}{HTML}{6C7680}
\definecolor{variable}{HTML}{001080}
\definecolor{string}{HTML}{c18100}
\definecolor{numbers}{HTML}{3f334d}
\ifx
\lstset{
	% How/what to match
	sensitive=true,
	% Border (above and below)
	frame=single,
	% Extra margin on line (align with paragraph)
	xleftmargin=\parindent,
	% Put extra space under caption
	belowcaptionskip=1\baselineskip,

	% Break long lines into multiple lines?
	breaklines=true,
	% Show a character for spaces?
	showstringspaces=false,
	tabsize=2
}
\fi
%END LISTINGDEF
\lstdefinestyle{lectureNotesListing}{
    numbers=left,
    xleftmargin=0.8em, % 2.8 with line numbers
    breaklines=true,
    frame=single,
    framesep=0.6mm,
    frameround=ffff,
    framexleftmargin=0.4em, % 2.4 with line numbers | 0.4 without them
    tabsize=4, %width of tabs
    aboveskip=1.0em,
    classoffset=0,
    sensitive=true,
	% Colors
backgroundcolor=\color{background},
basicstyle=\color{basicStyle}\small\ttfamily,
keywordstyle=\color{keywords},
commentstyle=\color{comment},
stringstyle=\color{string},
numberstyle=\color{numbers},
identifierstyle=\color{variable},
    showstringspaces=true
}
\lstset{style=lectureNotesListing}
%END LISTINGDEF



% variables you can adjust
\newcommand{\titleVar}{EIGEN Key Methods / Points}
\newcommand{\authorVar}{David Bohner}
%\newcommand{\dateVar}{\today}
\newcommand{\dateVar}{HS 2019}
%\newcommand{\rHeadVar}{\titleVar}
\newcommand{\rHeadVar}{EIGEN Summary}
\newcommand{\lHeadVar}{\authorVar}
% end variables you can adjust

% last configuration before document starts
\author{\authorVar}
\title{\titleVar}
\date{\dateVar}


%making nice headers & footers
\pagestyle{fancy}
\fancyhf{}
\rhead{\rHeadVar}
\lhead{\lHeadVar}
\rfoot{Page \thepage}
% ending last configuration


% optional math environments
\newtheorem{defn}{Definition}[section]
% end optional math environments
    
\begin{document}


\maketitle

\begin{lstlisting}[language=c++]
#include <Eigen/Dense>
\end{lstlisting}

% if you dont want a table of contents delete the next line
\tocr

% this enables headers and footers on the first page
\thispagestyle{fancy}

\section{EIGEN matrix data types}
\begin{itemize}
\item \textbf{MatrixXd}: Generic, variable-sized matrix with \textbf{d}ouble precision entries.
\item \textbf{VectorXi}, \textbf{RowVectorXf}: Generic variable-sized column and row vectors, with \textbf{i}nt / \textbf{f}loat precision entries.
\item \textbf{MatrixNcd}, \textbf{VectorNd} with $N$ = 2, 3, ...: A fixed-size $N$x$N$ matrix/vector. Matrices/Vectors whose dimensions are known at compile-time are generally better optimized in terms of memory/storage requirements. \textbf{cd} stands for complex double.
\end{itemize}
\begin{lstlisting}[language=c++]
Eigen::MatrixXd A = Eigen::MatrixXd::Constant(n, m, 0);
\end{lstlisting}
\section{Special Matrices}
\begin{lstlisting}[language=c++]
Eigen::MatrixXd I = Eigen::MatrixXd::Identity(n,n);
Eigen::MatrixXd O = Eigen::MatrixXd::Zero(n,m);
Eigen::MatrixXd D = d_vector.asDiagonal();
\end{lstlisting}
\subsection{Sparse Matrices}
\begin{lstlisting}[language=c++]
#include <Eigen/Sparse>
//CSS Format
Eigen::SparseMatrix<int, Eigen::ColMajor> Asp(rows, cols);
//CRS Format
Eigen::SparseMatrix<int, Eigen::RowMajor> Asp(rows, cols);

unsigned int row_ind = 2;
unsigned int col_ind = 4;
double value = 2.5;
Eigen::Triplet<double> triplet(row_ind, col_ind, value);
Eigen::SparseMatrix<double, Eigen::RowMajor> spMat(rows, cols);
spMat.setFromTriplets(triplets.begin(), triplets.end());
\end{lstlisting}
\section{Reading Matrix Values}
\begin{lstlisting}[language=c++]
using namespace Eigen;

MatrixXd R = MatrixXd::Random(m,n); //Values between -1 and 1
int a = R.rows(); //a = m.
int b = R.cols(); //b = n.
R(i,j) = 2;
R.block(i,j,h,w); //Block size hxw starting at (i,j)
\end{lstlisting}
Other methods include: \texttt{topLeftCorner(p,q), bottomLeftCorner(p,q), topRightCorner(p,q), bottomRightCorner(p,q), topRows(q), bottomRows(q), leftCols(p), rightCols(p)}.
\section{Building Matrices}
\begin{lstlisting}[language=c++]
MatrixXd arrowmat(const VectorXd &d, const VectorXd &c, const VectorXd &b, const double alpha) {
	int n = d.size();
	MatrixXd A(n+1, n+1);	//Empty dense matrix
	A.setZero();			//Init with all zeroes
	A.diagonal.head(n) = d;	//Initialize diagonal from vector d
	A.col(n).head(n) = c;	//Rightmost column = c
	A.row(n).head(n) = b;	//Bottom row = b (b is a row vector)
	A(n, n) = alpha;
	return A;
}
\end{lstlisting}
\section{Basic Operations}
\subsection{Arithmetic}
\begin{lstlisting}[language=c++]
A + B; A - B; A*B; n*A;
A*x; x.adjoint()*A*x;
\end{lstlisting}
\subsection{Products}
\begin{lstlisting}[language=c++]
x.dot(y); //Dot product. x, y are column vectors.
x.adjoint()*y; //Dot product by explicitly transposing x.
x * y.adjoint(); //Tensor product.
\end{lstlisting}
\section{LSE Solvers}
\subsection{Gaussian Elimination}
The default solver in EIGEN for \textbf{matrix decompositions} and \textbf{solving LSE}s is Gaussian elimination with partial pivoting. This is accessible through the methods \texttt{lu()} and \texttt{solve()} on dense matrices.
\begin{lstlisting}[language=c++]
X = A.lu().solve(B)
\end{lstlisting}
Wherein $A \in \mathbb{K}^{nn}$, $B \in \mathbb{K}^{nj}$, and $B$ represents $j$ RHS vectors.\\
$X = A^{-1}B = \left[ A^{-1}(B)_{:,1}, ..., A^{-1}(B)_{:,l}\right]$
\subsubsection{Using Special Structures}
\begin{lstlisting}[language=c++]
//A is lower triangular
x = A.triangularView<Eigen::Lower>().solve(b);
//A is upper triangular
x = A.triangularView<Eigen::Upper>().solve(b);
//A is hermetian/self-adjoined and positive definite
x = A.selfadjointView<Eigen::Upper>().llt().solve(b);
//A is hermetian/self-adjoined, positive or negative semi-definite
x = A.selfadjointView<Eigen::Upper>().ldlt().solve(b);
\end{lstlisting}

\begin{lstlisting}[language=c++]
x = A.lu().solve(b); //lu short for partial piv lu
x = A.fullPivLu().solve(b);
Eigen::HouseholderQR<MatrixXd> solver(A); x = solver.solve(b);
x = A.jacobiSvd(Eigen::ComputeThinU|Eigen::ComputeThinV).solve(b);
\end{lstlisting}
\subsubsection{Sparse Matrices}
\begin{lstlisting}[language=c++]
Eigen::SolverType<Eigen::SparseMatrix<double>> solver(A);
Eigen::VectorXd x = solver.solve(b);
\end{lstlisting}
The standard solver is \texttt{SparseLU}.
\begin{lstlisting}[language=c++]
//SparseLU for an arrow matrix
VectorXd arrowsys_sparse(const VectorXd &d, const VectorXd &c, const VectorXd &b, const double alpha, const VectorXd &y) {
	int n = d.size();
	SparseMatrix<double> A(n+1, n+1); //default: ccs
	VectorXi reserveVec = VectorXi::Constant(n+1, 2); //nnz per col
	reserveVec(n) = n+1; //last full col
	A.reserve(reserveVec);
	for(int j=0; j<n; j++) { //initialize along cols for efficiency
		A.insert(j, j) = d(j); //diagonal entry
		A.insert(n, j) = b(j);
	}
	for(int i=0; i<n; i++) {
		A.insert(i, n) = c(i); //last col
	}
	A.insert(n,n) = alpha; //bottom right entry
	
	A.makeCompressed();
	return solver_t(A).solve(y);
}
\end{lstlisting}
Implementations of Sparse Solvers:
\begin{itemize}
\item SuperLU
\item UMFPACK
\item \textbf{PARDISO}
\end{itemize}
\begin{lstlisting}[language=c++]
#include <Eigen/PardisoSupport>
void solveSparsePardiso(size_t n) {
	using SpMat = Eigen::SparseMatrix<double>;
	
	const SpMat M = initSparseMatrix <SpMat>(n);
	const Eigen::VectorXd b = Eigen::VectorXd::Random(n);
	Eigen::VectorXd x(n);
	
	Eigen::PardisoLU<SpMat> solver(M);
	x = solver.solve(b);
}
\end{lstlisting}
\end{document}