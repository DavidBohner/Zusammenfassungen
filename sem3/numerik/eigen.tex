\documentclass[12pt]{article}

% all the packages we will import
\usepackage[dvipsnames,usenames, table]{xcolor} 
\usepackage[utf8]{inputenc}
\usepackage[T1]{fontenc}
\usepackage[english]{babel}
\usepackage[a4paper, margin=1in]{geometry}
\usepackage{lipsum}
\usepackage{libertine}
\usepackage{amsthm}
\usepackage{amssymb}
\usepackage{amsmath}
\usepackage{libertinust1math}
\usepackage{tikz}
\usepackage{mathtools}
\usepackage{fancyhdr}
\usepackage{tikz}
\usepackage{graphicx}
\usepackage{microtype}
\usepackage{listings}
\usepackage[hidelinks]{hyperref}
\usepackage{tocloft}
\usepackage{lettrine}
\usepackage{GoudyIn}

% commands we are defining / redefining
\renewcommand{\LettrineFontHook}{\GoudyInfamily{}}

\newcommand{\tocr}{
\renewcommand\cftaftertoctitle{\par\noindent\hrulefill\par\vskip-0.65em}
\tableofcontents
\noindent\hrulefill
}
\DeclarePairedDelimiter{\ceil}{\lceil}{\rceil}
\renewcommand{\ttdefault}{pcr}

    
        
\usepackage{inconsolata}

\definecolor{background}{HTML}{fcfeff}
\definecolor{comment}{HTML}{b2979b}
\definecolor{keywords}{HTML}{255957}
\definecolor{basicStyle}{HTML}{6C7680}
\definecolor{variable}{HTML}{001080}
\definecolor{string}{HTML}{c18100}
\definecolor{numbers}{HTML}{3f334d}
\ifx
\lstset{
	% How/what to match
	sensitive=true,
	% Border (above and below)
	frame=single,
	% Extra margin on line (align with paragraph)
	xleftmargin=\parindent,
	% Put extra space under caption
	belowcaptionskip=1\baselineskip,

	% Break long lines into multiple lines?
	breaklines=true,
	% Show a character for spaces?
	showstringspaces=false,
	tabsize=2
}
\fi
%END LISTINGDEF
\lstdefinestyle{lectureNotesListing}{
    numbers=left,
    xleftmargin=0.8em, % 2.8 with line numbers
    breaklines=true,
    frame=single,
    framesep=0.6mm,
    frameround=ffff,
    framexleftmargin=0.4em, % 2.4 with line numbers | 0.4 without them
    tabsize=4, %width of tabs
    aboveskip=1.0em,
    classoffset=0,
    sensitive=true,
	% Colors
backgroundcolor=\color{background},
basicstyle=\color{basicStyle}\small\ttfamily,
keywordstyle=\color{keywords},
commentstyle=\color{comment},
stringstyle=\color{string},
numberstyle=\color{numbers},
identifierstyle=\color{variable},
    showstringspaces=true
}
\lstset{style=lectureNotesListing}
%END LISTINGDEF



% variables you can adjust
\newcommand{\titleVar}{EIGEN Key Methods / Points}
\newcommand{\authorVar}{David Bohner}
%\newcommand{\dateVar}{\today}
\newcommand{\dateVar}{HS 2019}
%\newcommand{\rHeadVar}{\titleVar}
\newcommand{\rHeadVar}{EIGEN Summary}
\newcommand{\lHeadVar}{\authorVar}
% end variables you can adjust

% last configuration before document starts
\author{\authorVar}
\title{\titleVar}
\date{\dateVar}


%making nice headers & footers
\pagestyle{fancy}
\fancyhf{}
\rhead{\rHeadVar}
\lhead{\lHeadVar}
\rfoot{Page \thepage}
% ending last configuration


% optional math environments
\newtheorem{defn}{Definition}[section]
% end optional math environments
    
\begin{document}


\maketitle

\begin{lstlisting}[language=c++]
#include <Eigen/Dense>
\end{lstlisting}

% if you dont want a table of contents delete the next line
\tocr

% this enables headers and footers on the first page
\thispagestyle{fancy}

\section{EIGEN matrix data types}
\begin{itemize}
\item \textbf{MatrixXd}: Generic, variable-sized matrix with \textbf{d}ouble precision entries.
\item \textbf{VectorXi}, \textbf{RowVectorXf}: Generic variable-sized column and row vectors, with \textbf{i}nt / \textbf{f}loat precision entries.
\item \textbf{MatrixNcd}, \textbf{VectorNd} with $N$ = 2, 3, ...: A fixed-size $N$x$N$ matrix/vector. Matrices/Vectors whose dimensions are known at compile-time are generally better optimized in terms of memory/storage requirements. \textbf{cd} stands for complex double.
\end{itemize}
\begin{lstlisting}[language=c++]
Eigen::MatrixXd A = Eigen::MatrixXd::Constant(n, m, 0);
\end{lstlisting}
\section{Special Matrices}
\begin{lstlisting}[language=c++]
Eigen::MatrixXd I = Eigen::MatrixXd::Identity(n,n);
Eigen::MatrixXd O = Eigen::MatrixXd::Zero(n,m);
Eigen::MatrixXd D = d_vector.asDiagonal();
\end{lstlisting}
\section{Reading Matrix Values}
\begin{lstlisting}[language=c++]
using namespace Eigen;

MatrixXd R = MatrixXd::Random(m,n); //Values between -1 and 1
int a = R.rows(); //a = m.
int b = R.cols(); //b = n.
R(i,j) = 2;
R.block(i,j,h,w); //Block size hxw starting at (i,j)
\end{lstlisting}
Other methods include: \texttt{topLeftCorner(p,q), bottomLeftCorner(p,q), topRightCorner(p,q), bottomRightCorner(p,q), topRows(q), bottomRows(q), leftCols(p), rightCols(p)}.
\section{Basic Operations}
\subsection{Arithmetic}
\begin{lstlisting}[language=c++]
A + B; A - B; A*B; n*A;
A*x; x.adjoint()*A*x;
\end{lstlisting}
\subsection{Products}
\begin{lstlisting}[language=c++]
x.dot(y); //Dot product. x, y are column vectors.
x.adjoint()*y; //Dot product by explicitly transposing x.
x * y.adjoint(); //Tensor product.
\end{lstlisting}
\end{document}