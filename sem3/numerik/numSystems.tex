\documentclass[12pt]{article}

% all the packages we will import
\usepackage[dvipsnames,usenames, table]{xcolor} 
\usepackage[utf8]{inputenc}
\usepackage[T1]{fontenc}
\usepackage[english]{babel}
\usepackage[a4paper, margin=1in]{geometry}
\usepackage{lipsum}
\usepackage{libertine}
\usepackage{amsthm}
\usepackage{amssymb}
\usepackage{amsmath}
\usepackage{libertinust1math}
\usepackage{tikz}
\usepackage{mathtools}
\usepackage{fancyhdr}
\usepackage{tikz}
\usepackage{graphicx}
\usepackage{microtype}
\usepackage{listings}
\usepackage[hidelinks]{hyperref}
\usepackage{tocloft}
\usepackage{lettrine}
\usepackage{GoudyIn}
\usepackage{tabularx}

% commands we are defining / redefining
\renewcommand{\LettrineFontHook}{\GoudyInfamily{}}

\newcommand{\tocr}{
\renewcommand\cftaftertoctitle{\par\noindent\hrulefill\par\vskip-0.65em}
\tableofcontents
\noindent\hrulefill
}
\DeclarePairedDelimiter{\ceil}{\lceil}{\rceil}
\renewcommand{\ttdefault}{pcr}

    
        
\usepackage{inconsolata}

\definecolor{background}{HTML}{fcfeff}
\definecolor{comment}{HTML}{b2979b}
\definecolor{keywords}{HTML}{255957}
\definecolor{basicStyle}{HTML}{6C7680}
\definecolor{variable}{HTML}{001080}
\definecolor{string}{HTML}{c18100}
\definecolor{numbers}{HTML}{3f334d}
\ifx
\lstset{
	% How/what to match
	sensitive=true,
	% Border (above and below)
	frame=single,
	% Extra margin on line (align with paragraph)
	xleftmargin=\parindent,
	% Put extra space under caption
	belowcaptionskip=1\baselineskip,

	% Break long lines into multiple lines?
	breaklines=true,
	% Show a character for spaces?
	showstringspaces=false,
	tabsize=2
}
\fi
%END LISTINGDEF
\lstdefinestyle{lectureNotesListing}{
    numbers=left,
    xleftmargin=0.8em, % 2.8 with line numbers
    breaklines=true,
    frame=single,
    framesep=0.6mm,
    frameround=ffff,
    framexleftmargin=0.4em, % 2.4 with line numbers | 0.4 without them
    tabsize=4, %width of tabs
    aboveskip=1.0em,
    classoffset=0,
    sensitive=true,
	% Colors
backgroundcolor=\color{background},
basicstyle=\color{basicStyle}\small\ttfamily,
keywordstyle=\color{keywords},
commentstyle=\color{comment},
stringstyle=\color{string},
numberstyle=\color{numbers},
identifierstyle=\color{variable},
    showstringspaces=true
}
\lstset{style=lectureNotesListing}
%END LISTINGDEF



% variables you can adjust
\newcommand{\titleVar}{Numerical Methods for CSE\\Summary}
\newcommand{\authorVar}{David Bohner}
%\newcommand{\dateVar}{\today}
\newcommand{\dateVar}{HS 2019}
%\newcommand{\rHeadVar}{\titleVar}
\newcommand{\rHeadVar}{NumCSE Summary}
\newcommand{\lHeadVar}{\authorVar}
% end variables you can adjust


% last configuration before document starts
\author{\authorVar}
\title{\titleVar}
\date{\dateVar}


%making nice headers & footers
\pagestyle{fancy}
\fancyhf{}
\rhead{\rHeadVar}
\lhead{\lHeadVar}
\rfoot{Page \thepage}
% ending last configuration


% optional math environments
\newtheorem{defn}{Definition}[section]
% end optional math environments
    
\begin{document}
    
    
\maketitle

% if you dont want a table of contents delete the next line
\tocr

% this enables headers and footers on the first page
\thispagestyle{fancy}
\section*{Pre-amble} %TODO
This document does not contain knowledge from, and expects an understanding of, any content in a Linear Algebra lecture.
\section{Computing with Matrices and Vectors}
\subsection{Fundamentals}
\subsubsection{Notations}
The default format of \textit{\textbf{vectors}} in this lecture is a \textit{\textbf{column vector}}.\\
The \textit{\textbf{Kronecker symbol}} $\delta_{ij} := 1$ if $i=j$, 0 otherwise.\\
An \textit{\textbf{Adjoined Matrix}} is \textit{\textbf{Hermetian-Transposed}}.
\subsubsection{Classes of Matrices}
Special Matrices:
\begin{itemize}
\item Identity Matrix
\item Zero Matrix
\item Diagonal Matrix
\item Upper [Lower] Triangular Matrix
\end{itemize}
\textit{\textbf{Symmetric (Hermetian) Positive Definite}} (spd) Matrices:
\begin{itemize}
\item $M = M^H$
\item $\forall x \in \mathbb{K}^n: x^HMx > 0 \iff x \neq 0$
\item $\forall 1 \leq i \leq n : m_{ii} > 0$
\item $\forall 1 \leq i < j \leq n: m_{ii}m_{jj} - |m_{ij}|^2 > 0$
\end{itemize}
If a Matrix has $\forall x \in \mathbb{K}^n: x^HMx \geq 0$, it is \textit{\textbf{positive semi-definite}}.
\subsection{Software and Libraries}
\subsubsection{EIGEN}
EIGEN is a \textit{header-only} C++ template library. It provides data structures and operations on them for matrices and vectors, as well as some more fundamental algorithms.\\% It has an incomplete and/or outdated documentation.
EIGEN matrix data types:
\begin{itemize}
\item \textbf{MatrixXd}: Generic, variable-sized matrix with \textbf{d}ouble precision entries.
\item \textbf{VectorXi}, \textbf{RowVectorXf}: Generic variable-sized column and row vectors, with \textbf{i}nt / \textbf{f}loat precision entries.
\item \textbf{MatrixNcd}, \textbf{VectorNd} with $N$ = 2, 3, ...: A fixed-size $N$x$N$ matrix/vector. Matrices/Vectors whose dimensions are known at compile-time are generally better optimized in terms of memory/storage requirements. \textbf{cd} stands for complex double.
\end{itemize}

\subsubsection{Dense Matrix Storage Formats}
\subsection{Basic Linear Algebra Operations}
\subsubsection{Elementary Matrix-Vector Calculus}
\subsection{Computational Effort}
\subsubsection{Asymptotic Complexity}
\subsubsection{Cost of Basic Linear-Algebra Operations}
\subsubsection{Reducing Complexity in Numerical Linear Algebra}
\subsection{Machine Arithmetic and Consequences}
\subsubsection{Loss of Orthogonality}
\subsubsection{Roundoff Errors}
\subsubsection{Cancellation}

\section{Direct Methods for Square LSE}
\subsection{Elimination Solvers for LSE}
\subsection{Exploiting Structure when Solving LSE}
\subsection{Sparse Linear Systems}
\subsubsection{Sparse Matrix Storage Formats}
\subsubsection{Sparse Matrices in EIGEN}
\subsubsection{Direct Solution of Sparce LSE}

\section{Direct Methods for Linear Least Squares Problems}
\subsection{Least Squares Solution Concepts}
\subsubsection{Least Squares Solutions}
\subsubsection{Normal Equations}
\subsubsection{Moore-Penrose Pseudoinverse}
\subsection{Normal Equation Methods}
\end{document}
    
    