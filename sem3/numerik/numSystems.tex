\documentclass[12pt,a4paper]{article}
\usepackage[utf8]{inputenc}
\usepackage{amsmath}
\usepackage{amsfonts}
\usepackage{amssymb}
%\usepackage{tikz}
%\usepackage{graphicx}
%\usetikzlibrary{positioning}
\usepackage{tabularx}

\usepackage{hyperref}
\usepackage{color}
\hypersetup{
	colorlinks,
	filecolor=black,
	linkcolor=black,
	urlcolor=black
}
\usepackage{listings}
%\usepackage[table]{xcolor}
%\setlength{\parindent}{0cm}

%TODO : 

\author{David Bohner}
\title{Numerical Methods for CSE \\ Summary}
\date{HS 2019}

\begin{document}
\maketitle
\tableofcontents
\newpage
\section{Definitions}
\begin{tabularx}{\linewidth}{r X}
Matrix \textbf{entry} & $(A)_{i,j}$\\
Matrix \textbf{block} & $(A)_{k:l,r:s}: 1 \leq k \leq l \leq n, 1 \leq r \leq s \leq m$\\
\textbf{Tensor} product & Vector $\times$ Vector ($xy^H$)
\end{tabularx}

\section{Software and Libraries}
\subsection{Eigen}
A C++ LinAlg library. Fundamental data type: Matrices.\\

\begin{tabularx}{\linewidth}{r X}
\textbf{Fixed} matrices: & n, m known at compile time.\\ & Optimized well by compiler.\\
\textbf{Variable} matrices: & n, m unknown at compile.\\ & Memory usage unoptimized.
\end{tabularx}

\subsection{Matrix Storage Formats}
$A := \begin{bmatrix} 1 & 2 \\ 3 & 4 \end{bmatrix}$ \\
\begin{tabularx}{\linewidth}{r X}
\textbf{Row} Major & A\_arr = \{1, 2, 3, 4\}\\
& (\textbf{C-arrays}, Bitmaps, Python, ...)\\
\textbf{Column} Major & A\_arr = \{1, 3, 2, 4\}\\
& (\textbf{Eigen}, Fortran, Matlab, ...)
\end{tabularx}\\\\\\
Data layout can be important for \textbf{performance};\\
\textbf{Computational effort} (asymptotic complexity) $\neq$ \textbf{Runtime}.\\
Asymptotic complexity allows us to predict the \textbf{increase in runtime} depending on increase in data size.
\subsubsection{Example:}
\begin{itemize}
\item[Q:] What is the optimal storage format for a Matrix $\times$ Vector calculation?
\item[A:] Row major storage. (Consider the optimal order of computation.)
\end{itemize}
\subsubsection{Cost of Basic Matrix-Vector Ops}
\begin{tabularx}{\linewidth}{c c c c c}
operation & description & \#mul/div & \#add/sub & asym. complex.\\
dot product & $(x \in \mathbb{R}^n, y \in \mathbb{R}^n) \mapsto x^Hy$ & $n$ & $n-1$ & $O(n)$\\
tensor product & $(x \in \mathbb{R}^m, y \in \mathbb{R}^n) \mapsto xy^H$ & $mn$ & $0$ & $O(mn)$\\
matrix product* & $(A \in \mathbb{R}^{m,n}, B \in \mathbb{R}^{n,k}) \mapsto AB$ & $mnk$ & $mk(n-1)$ & $O(mnk)$
\end{tabularx}
*triple-loop implementation
\end{document}