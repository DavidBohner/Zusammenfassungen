\documentclass[12pt,a4paper]{article}
\usepackage[utf8]{inputenc}
\usepackage{amsmath}
\usepackage{amsfonts}
\usepackage{amssymb}
%\usepackage{tikz}
%\usepackage{graphicx}
%\usetikzlibrary{positioning}
\usepackage{tabularx}

\usepackage{hyperref}
\usepackage{color}
\hypersetup{
	colorlinks,
	filecolor=black,
	linkcolor=black,
	urlcolor=black
}
\usepackage{listings}
%\usepackage[table]{xcolor}
%\setlength{\parindent}{0cm}

%TODO : 

\author{David Bohner}
\title{Analysis II \\ Summary}
\date{HS 2019}

\begin{document}
\maketitle
\tableofcontents
\newpage
\section{Definitions}
\begin{tabularx}{\linewidth}{r X}
Differential Equation & The unknown/s is/are a \textbf{function} $\mathbf{f}$, and relates values of $f$ at a point $x$ with values of derivatives of $f$ at $x$.\\
Ordinary DE & A differential equation where the function $f$ has one variable only.\\
\textbf{Order} (of a DE) & The \textbf{highest} derivative that appears in a diff. equation\\
\textbf{Linear} DE & See below.\\
\textbf{Homogenous} LDE & See below.
\end{tabularx}

\subsection{Examples of Differential Equations}
\begin{tabular}{|c|c|c|} \hline
$f' = f$ & $f'(x+1) = f(x)$ & $f'(x)^2 = f(x)^3 - f(x)$\\
One sol: $f(x) = \alpha e^x$ & \textbf{Not ordinary} & Or: $e^x f'(x)^2 = x f(x)^3 - f(x) + 1$\\ \hline
$f'=a$ & All laws of physics\\
a being a \textbf{fixed} function of x & Well, at least \textit{almost all}\\ \hline
\end{tabular}

\subsection{Ordinary Differential Equations}
Differential equations are \textbf{not} ordinary if they relate the value of $f$ at a point $x$ with the derivative of \textbf{another point}, like $f'(x+1) - f(x) = 0$.

\subsection{Linear Differential Equations}
Equations of the form: $y^{(k)} + a_{k-1}y^{(k-1)} + ... + a_1y' + a_0y = b$ where the coefficients $a_i$ are complex-valued \textit{functions} on $I$ (an interval), and the unknown is a complex-valued function from $I$ to $C$ that is k-times differentiable on $I$, are \textit{linear differential equations}.\\
In the case of $b = 0$, the equation is an \textit{homogeneous linear ordinary differential equation}.
In the case of $b$ being a function $b: I \to C$, it is an \textit{inhomogeneous linear ordinary differential equation}, with \textit{associated homogeneous equation} the one with $b = 0$.

\end{document}